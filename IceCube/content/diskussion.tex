\section{Discussion}
\label{sec:Discussion}

To classify 4000 given events as signal or background, a Naive Bayes classifier, a Random Forest classifier as well as a kNN classifier were 
trained using the ten most relevant features of the dataset according to a mRMR selection. The performance of the classifiers was evaluated using
the area under the ROC curve $A_{\mathrm{ROC}}$, the accuracy and the precision. The final values of the $A_{\mathrm{ROC}}$ for the classifiers are
\begin{align*}
    A_{\mathrm{ROC, Bayes}} &= \num{0.9400+-0.0023} \\
    A_{\mathrm{ROC, RF}} &= \num{0.9829(0.0012)} \\
    A_{\mathrm{ROC, kNN}} &= \num{0.9211 \pm 0.0033}.
\end{align*}
\\While all three classifiers yield very good results, the Random Forest performs best on the test subset. Therefore, this model is chosen for the 
final prediction of the labels.\\
To achieve even higher scores for the quality of the predictions, one can optimize the number of input features and in general put more
effort into data preparation before training the model. This can be done by scaling the input parameters to achieve a better distinction 
between the distributions of the used features.\\
According to the confusion matrices of the classifiers, most misclassified events are background events being classified as signal. This is
problematic, since this could lead to an overestimation of signal and therefore potential false discoveries.\\
Futher analysis and optimization for the trained models is needed, to achieve better performance in this classification problem and prevent
the problem of many background events being classified as signal.

% - Optimize number of features
% - Context confusion matrix and f_beta score ?
% - in general no real optimization done