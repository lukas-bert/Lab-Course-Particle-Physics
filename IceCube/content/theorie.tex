\section{Motivation}
The study of particles from astrophysical sources is of great interest to understand stellar and galactic processes, as well as our universe itself.
Neutrinos are well suited for these studies, since they are uncharged and only interact weakly, allowing to trace back measured neutrinos on earth to their origin
in far away astrophysical sources. \\
In this analysis, a selection of neutrino events is performed using Monte Carlo simulation data from the IceCube experiment. 
A minimum redundancy, maximum relevance (\textit{mRMR}) selection is employed to determine the most suitable features for a multivariate analysis 
separating signal and background events. Three different machine algorithms are compared and their performance on the classification task is evaluated.


\section{Theory}
\label{sec:Theory}
In this section, the theory aspects of the analysis are explained.
At first, the basics of astroparticle physics and neutrino detections are described and then, machine learning methods utilized in this study are discussed. 

\subsection{Fundamentals of Astroparticle Physics and Cosmic Rays}
The earth is constantly hit by ionizing, high energy particles originating from astrophysical sources in the universe which are called \textit{cosmic rays}. 
The majority of these particles are protons, light and heavy nuclei and electrons that interact with the earth's atmosphere and cause large cascades of particle decays 
(\enquote{\textit{air showers}})
which can be measured as cosmic radiation on earth. Since these particles are charged, they are deflected by galactic and extragalactic magnetic fields on their way to earth and 
cannot be traced back to their origin. The energy spectrum of these charged particles extends up to $10^{20}\, \unit{\mega\eV}$ and has a flux described by the power law 
\begin{equation}
    \frac{\mathrm{d}\Phi}{\mathrm{d}E} = \Phi_0 E^\gamma,
    \label{eq:flux}
\end{equation}
where $\gamma \approx \num{-2.7}$ is the spectral index.
Apart from the charged particles, high energy gamma rays and neutrinos from outer space reach the earth. Since these particles are not charged, they can in theory be 
traced back to their origin. Neutrinos only interact via the weak interaction and have very small cross sections, allowing them to penetrate dense regions in the universe on 
their way to earth which is not possible for gamma rays that interact via the electromagnetic interaction.
This analysis will focus on neutrinos that are measured by the IceCube experiment, which is described in detail in \autoref{sec:Detector}.
At IceCube, atmospheric and cosmics neutrinos are measured. The atmospheric neutrinos are created in the previously mentioned air showers and are further categorized into 
conventional and prompt neutrinos. Conventional neutrinos originate from kaon and pion decays into a muon and muon neutrino. Since these particles have a comparatively long
lifetime, they lose a significant amount of energy before they decay, resulting in an energy spectrum with a spectral index of $\gamma \approx \num{-3.7}$. 
Prompt neutrinos stem from semi-leptonic decays of heavy hadrons like $D$ mesons and $\Lambda$ baryons which have a short lifetime and therefore do not lose as much energy,
resulting in a energy spectrum with $\gamma \approx \num{-2.7}$, similar to the spectrum of charged cosmic rays. Here, the cosmics neutrinos from astrophysical sources are the wanted signal to be measured. 
Under the assumption of shock acceleration \cite{Fermi1949}, the flux of these neutrinos has a spectral index of $\gamma \approx \num{-2}$.

\subsubsection{Measurement of neutrinos with the IceCube detector}
In IceCube, neutrinos are measured via \textit{Cherenkov light} of secondary particles that are created in interactions with the ice molecules.
Cherenkov light is emitted when a charged particle traverses a medium with a higher velocity than the respective speed of light in the medium.
The speed of light in a medium  with refractive index $n$ is given by $c = \sfrac{c_0}{n}$, where $c_0$ is the speed of light in vacuum.
The neutrinos can interact via the charged current (\textit{CC})
\begin{equation}
    \nu_l(\bar{\nu}_l) + A \to l^\mp + X
    \label{eq:CC}
\end{equation}
and the neutral current (\textit{NC})
\begin{equation}
    \nu_l + A \to \nu_l + X,
    \label{eq:NC}
\end{equation}
where $A$ are the ice nuclei and $X$ all other finals state particles in the reaction. Different signatures of the events allow to distinguish between 
different lepton flavors in \autoref{eq:CC} and NC events. 
Electrons and NC events create circular, cascade like events, whereas muons create long tracks in the detector. Tau leptons have a short lifetime and therefore 
cause events with two nearby circular cascades. Further information on the IceCube detector system is given in \autoref{sec:Detector}.

\subsection{Feature Selection and Multivariate Analysis}
In this section, the feature selection and machine learning algorithms that are employed in this analysis are briefly described.
\subsubsection{mRMR Selection}
In a mRMR (minimum redundancy, maximum relevance) selection, a set of variables (features) is iteratively selected that strongly correlates with the target (signal or background)
and has a low redundancy (correlation between features). For this purpose, the joint information of two variables $x, y$
\begin{equation}
    I(x, y) = \int p(x,y) \symup{log}\left(\frac{p(x,y)}{p(x)p(y)}\right)\mathrm{d}x\mathrm{d}y
    \label{eq:mRMR}
\end{equation}
is considered, where $p(x/y)$ are the respective probability functions. Here, the mRMR implementation in the python package \textit{mrmr-selection} \cite{mrmr} is used.

\subsubsection{Na\"ive Bayes Classifier}
In a na\"ive bayes classifier, Bayes theorem on conditional probabilites
\begin{equation}
    p(A|B) = \frac{p(B|A)p(A)}{p(B)}
    \label{eq:Bayes}
\end{equation}
is used to express the likelihood of an event belonging to a class $A(\overline{A})$ (signal (background)) using features $B_i$.
With $n$ attributes, the measure 
\begin{equation}
    Q = \prod_{i=1}^{n} \frac{p(B_i|A)}{p(B_i|\overline{A})}
\end{equation}
is used to distinguish between signal with $Q > 1$ and background.

\subsubsection{Decision Trees}
Decision trees are binary classifiers, that separate between different classes by subsequently applying binary cuts on the available variables. 
For each decision point (\textit{node}), the cut that maximes the separation between the classes is searched. The data is divided into two subsets after each cut, till a
maximum number of cuts (depth) is reached or the classes are fully separated.
To minimize overtraining, different techniques can be employed to make the classification more robust. For example, a boosted decision tree (\textit{BDT}) can be used.
A BDT is an ensemble of multiple decision trees, where each tree is sequentially trained with weighting previously misclassified data higher. By using this approach, 
the bias and variance of the classification are reduced, while the accuray is inreased.

\subsubsection{Some other MVA method that is used here}
Schreibe ich noch, je nachdem was Tom noch für classifier nimmt.

\subsubsection{Evaluation Metrics}
In order to evaluate the performance of a classifier on the classification task, different metrics are used.
The classification output can be grouped into four categories: Correctly classified signal events (true positives \enquote{\textit{tp}}), correctly classified background 
(true negatives \enquote{\textit{tn}}), background events falsely classified as signal (false positives \enquote{\textit{fp}}) and signal events falsely classified as background
(false negatives \enquote{\textit{fn}}). With these definition, the accuracy 
\begin{equation}
    a = \frac{\text{tp} + \text{tn}}{\text{tp} + \text{tn} + \text{fn} + \text{fp}},
    \label{eq:accuracy}
\end{equation}
the precision
\begin{equation}
    p = \frac{\text{tp} }{\text{tp} + \text{fp}},
    \label{eq:precision}
\end{equation}
and the recall 
\begin{equation}
    r = \frac{\text{tp} }{\text{tp} + \text{fn}}
    \label{eq:recall}
\end{equation}
can be defined to evaluate the performance of the classification.
Another useful metric is the $f_\beta$ score
\begin{equation}
    f_\beta = (1 + \beta^2)\frac{p \cdot r}{\beta^2 p + r},
    \label{eq:f_beta}
\end{equation}
which is the harmonic mean of precision and recall with the recall weighed by a factor $\beta$.
These metrics depend on the classification threshold $t$ that is applied to separate signal and background by a cut on the classifiers output, which is typically a value 
between $0$ and $1$, where $1$ indicates a high probability of being signal and vice versa.
A metric independent of the threshold $t$ can be obtained by the \textit{Receiver Operating Characteristic} (ROC) curve. In the ROC curve, the true positive rate 
(\textit{TPR}) is plotted against the false positve rate (\textit{FPR}), with the TPR and FPR defined as 
\begin{align}
    \text{TPR}(t) &= \frac{\text{tp}(t)}{\text{tp}(t) + \text{fn}(t)} & \text{FPR}(t) &= \frac{\text{fp}(t)}{\text{fp}(t) + \text{tn}(t)},
    \label{eq:TPR_FPR}
\end{align}
while varying the threshold $t$. The area under the curve (AUC-score) is a measure for the quality of the classification and takes the values $\num{0.5}$ for random guessing 
and $\num{1}$ for a perfect classification. A AUC-score smaller than $\num{0.5}$ indicates, that the model confuses signal and background, which can easily be corrected by 
inverting the classification.
