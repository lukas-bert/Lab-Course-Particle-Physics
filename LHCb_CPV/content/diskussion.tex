\section{Discussion}
\label{sec:Discussion}
In this analysis, the global \textit{CP} asymmetry for the decays $B^\pm \to K^\pm K^+ K^-$ is determined as $A_{CP, \text{global}} = \num{999999}$ with a significance 
of $S_{A_{CP, \text{global}}} = \num{99999}$. In the 2013 LHCb analysis on the same data, the global \textit{CP} asymmetry is determined as 
\sisetup{uncertainty-descriptors = {sys, stat}}
\begin{equation*}
    {A_{CP}(B^\pm \to K^\pm K^+K^-)= \num{-0.04399999(0.009999)(0.00399999)}\pm \num{0.00999997}(JψK^\pm)}
\end{equation*}
with a significance of $\num{3.7}\sigma$ \cite{LHCb:2013ptu}. 
The opposite sign of the asymmetry is due to an opposite definition of the asymmetry observable in \autoref{eq:CP_asymmetry}. Considering the uncertainty of the \textit{CP} asymmetry 
determined in this analysis, the values are in agreement with each other. Although systematic uncertainties apart from the estimated production asymmetry are neglected in this 
analysis but are included in the LHCb analysis, the uncertainties of both values are comparable. This can be explained by the fact,
that the analysis here is strongly simplified and selection criteria are not optimised as much as in the LHCb paper. \\
From the Dalitz plots of the two-Kaon masses, resonances at approximately $\qty{1870}{\mega\eV}$ and $\qty{3390}{\mega\eV}$ are observed and 
can be identified with $D^0$ and $\chi_{c0}$ contributions. In the original paper, only the $D^0$ resonance is accounted for and removed.
The strongest local \textit{CP} asymmetry here is observed in low $m(KK)$ regions which is in agreement to the observations in Ref. \cite{LHCb:2013ptu}. 
The strongest local \textit{CP} asymmetry follows as ${A_{CP, \text{loc}}=\num{0.1129 +- 0.0195}}$ with a significance of $\num{5.2}\sigma$. In the LHCb analysis, the \textit{CP} asymmetry in 
low $m(KK)$ regions reads \sisetup{uncertainty-descriptors = {sys, stat}}
\begin{equation*}
    {A^\text{reg}_{CP}(B^\pm \to K^\pm K^+K^-) = \num{-0.226(0.020)(0.004)} \pm \num{0.007}(JψK^\pm)} \; \text{\cite{LHCb:2013ptu}.}
\end{equation*}
Here, the values are not in agreement, considering the given uncertainties. However, the uncertainty of the value from this analysis is clearly underestimated since no systematic 
uncertainties are included apart from the estimated production asymmetry of $\qty{1}{\percent}$. Following from that, the claimed significance of $\num{5.2}\sigma$ is 
overestimated. Further, the differences in the observed values could originate from different binning or different mass criteria of the low mass regions as 
defined in the LHCb paper and here. \\
All in all, the results achieved in this simplified \textit{CP} violation analysis are in good agreement with the LHCb observations on the same data. The neglection of systematic uncertainties and 
different selection criteria can explain the deviations of the observed asymmetries.
