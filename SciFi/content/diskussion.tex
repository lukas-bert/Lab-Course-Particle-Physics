\section{Diskussion}
\label{sec:Diskussion}
Die Spektrometermessung für eingeschaltetes und ausgeschaltetes Raumlicht aus \autoref{sec:Spektrometermessung} zeigt deutlich die Notwendigkeit der Abdunklung des
Raums. Hierbei ist zu beachten, dass die Bereinigung der Messwerte um die Dunkelzählrate erneut ein klareres Signal produziert. \\
Die radialsymmetrische Verteilung der Intensität der aus der Faser austretenden Photonen wird durch \autoref{fig:radial} bestätigt. Das Maximum der Intensität liegt nicht 
bei \qty{0}{\degree}. Es ist möglich, dass ein systematischer Offset in Ausrichtung der Messapparatur vorliegt. \\
Die bereitgestellten Simulationsdaten lassen sich in Kern- und Manteldaten unterteilen und die unterschiedlichen Wirkungen lassen sich in \autoref{fig:theta_sim}
deutlich erkennen.
Das verschiedene Verhalten wird nochmals in \autoref{fig:rmin_sim} verdeutlicht, wo der Winkel~$\theta$ gegenüber dem minimalen Abstand zur Fasermitte~$r_{\text{min}}$
aufgetragen wird. \\
Die Winkelabhängikeit der experimentell bestimmten Absorptionskoeffizienten steht in Übereinstimmung mit der theoretisch zu erwartenden (siehe \autoref{fig:absorptioncoefficient_data}).
Aus \autoref{fig:absorptioncoefficient_sim} wird jedoch ersichtlich, dass die Simulationsdaten keine Übereinstimmung mit der Theorie aufweisen. Es ist denkbar, dass
die Winkelabhängigkeit der Absorptionskoeffizienten in der Simulation nicht berücksichtigt wird. \\
Die Winkelintensitätsmessung liefert ein Maximum der Intensität für einen horizontalen Winkel von \qty{11}{\degree}. Auch hier ist es möglich, dass ein systematischer Offset
der Ausrichtung von dem Spektrometer die Daten verzerrt. In \autoref{fig:intensity_angle} wird auch ersichichtlich, dass die Winkel, die kleiner als \qty{11}{\degree} sind,
nur eine gerinfügig kleinere Intensität als das Maximum aufweisen. Eine längere Messdauer könnte gegenbenenfalls ein anderes Ergebnis liefern.
 