\section{Durchführung}
\label{sec:Durchführung}
Zur Analyse der Funktionsweise des SciFi Detektors wird in diesem Versuch eine einzelne szintillierende Faser genauer betrachtet. Der dafür verwendete 
Versuchsaufbau ist in \autoref{sec:aufbau} beschrieben und das Messprogramm wird in \autoref{sec:messprogramm} erklärt.

\subsection{Aufbau}
\label{sec:aufbau}
Eine szintillierende Faser ist entlang der x-Achse eines xy-Tisches eingespannt, sodass eine LED präzise über eine Länge von \qtyrange{0}{2000}{\milli\metre}
der Faser gefahren werden kann. An einem Ende der Faser sitzt ein Spektrometer, welches mit Schrittmotoren entlang der Horizontalen und der Vertikalen bewegt
werden kann. Ein angeschlossener Computer erlaubt die Steuerung der Motoren und der LED sowie die automatische Datenerfassung.

\subsection{Messprogramm}
\label{sec:messprogramm}
Bevor die Messungen starten können, muss der Raum abgedunkelt werden, damit Streulichteffekte unterdrückt werden können. Darüber hinaus wird bei jeder Messung 
sowohl die Anregung der Faser mittels der LED gemessen, als auch die Dunkelzählrate, sodass die Messdaten um diese bereinigt werden können. Für sämtliche Messungen
wird eine Integrationszeit von \qty{10000}{\micro\second} und einem Strom von \qty{20}{\milli\ampere} gewählt

\subsubsection{Spektrometermessung}
Zur Illustration der Auswirkungen von angeschaltetem Raumlicht wird eine Spektrometermessung bei eingeschaltetem und eine bei ausgeschaltetem Raumlicht durchgeführt.
Dazu wird an dem Steuerungscomputer in einer GUI eine Spektrometermessung
gestartet. Die Intensität der Messungen wird für die beiden Messreihen gegenüber der Wellenlänge aufgetragen. Außerdem wird in einem weiteren Schritt die 
Dunkelzählrate abgezogen und das Ergebnis erneut graphisch dargestellt.

\subsubsection{Radialsymmetrie}
Die Intensität des aus der Faser austretenden Lichts folgt einer Radialsymmetrie, welche in dieser Messung verifiziert werden soll. Dazu wird für eine feste Position
der LED auf der Faser das Spektrometer entlang der horizontalen und vertialen Achse verfahren. An dem Computer wird ein Programm aufgerufen, mit dem der aufzufahrende 
Bereich sowie die Schrittweite eingestellt werden kann. Der horizontale Winkelbereich soll von \qtyrange{-18}{30}{\degree} und der vertikale von \qtyrange{-6}{35}{\degree}
verlaufen.

\subsubsection{Simulation}
Neben den aufzunehmenden Messdaten werden ebenfalls Simulationsdaten analysiert. Dabei enthalten die Daten Informationen über die Anregungsorte der Faser, den Austrittsort
am Faserende, den Erzeugungsort des Photons mit dazugehörigem Impuls und Wellenlänge, die Anzahl der Reflexionen an der Core-Cladding und Cladding-Cladding Grenzfläche, die
Anzahl der Raleighstreuungen, sowie die zurückgelegten Wegstrecken in den verschiedenen Medien. \\
Zunächst werden unphysikalische Simulationseffekte bereinigt, indem alle Photonen, deren Austrittsort außerhalb des Faserradius liegt, entfernt werden. Außerdem 
werden alle Photonen, die Raleighstreuungen durchgeführt haben, ebenfalls entfernt. Der Datensatz wird anschließend in Kern- und Mantelphotonen aufgeteilt. Der Winkel~$\theta$
des Photons zur x-Achse wird berechnet und als neue Variable eingeführt. Die Intensität der verschiedenen Winkel~$\theta$ wird in einem Histogramm sowohl für die Kern- als auch 
für die Mantelphotonen aufgetragen. Der maximale Winkel unter dem noch Totalreflexion auftreten kann wird berechnet und ebenfalls im Histogramm eingezeichnet. \\
In einem nächsten Schritt wird der minimale Abstand der Kern- und Mantelphotonen zur Fasermitte berechnet und in einem zweidimensionalen Histogramm in Abhängigkeit zum Winkel~$\theta$
gesetzt. \\
Als nächstes wird die Intensität für verschiedene Winkel in Abhängigkeit zum Anregungsort auf der x-Achse bestimmt und abermals graphisch dargestellt. Mithilfe eines
exponentiellen Fits kann der Absorbtionskoeffizient der szintillierenden Fasern bestimmt werden.

\subsubsection{Intensitätsmessung}
In einem weiteren Schritt wird die x-Abhängige Intensität bestimmt. Dazu werden entlang der x-Achse 20 verschiedene Positionen der Faser angeregt und für jede dieser
Positionen der horizontale oder vertikale Winkel in zehn Schritten von \qtyrange{0}{40}{\degree} variiert. Diese Daten werden analog zu den Simulationsdaten ausgewerten und 
mit diesen verglichen.

\subsubsection{Winkelintensitätsmessung}
Zur Bestimmung des Winkels, bei dem die Intensität am größten ist, wird für einen festen Anregungsort und festen vertikalen Winkel der horizontale Winkel feinschrittig von
\qtyrange{0}{45}{\degree} verändert. Die Intensität wird gegen die Winkelverteilung histogrammiert und das Maximum dieser Verteilung bestimmt.