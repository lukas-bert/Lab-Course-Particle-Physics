\section{Analysis}
\label{sec:Analysis}
This section focuses on the analysis of the measured data. Because of a malfunctioning cable connection, many of the tasks could not be completed and for those
that data could be taken, the trustworthiness can be questioned. The data for all of the tasks, except for the measurement of the current-voltage characterisic,
was kindly provided by our supervisor.

\subsection{Current - voltage characteristic}
\label{sec:current-voltage characterisic}
In a first step, the sensor current $I$ is measured as a function of the applied bias voltage $U$. The resulting data are shown in \autoref{fig:U_dep}.
\begin{figure}
    \centering
    \includegraphics[width = 0.8\textwidth]{build/leakage.pdf}
    \caption{Plot of the measured current $I$ dependent on the applied voltage $U$. The depletion voltage $U_{\mathrm{dep}} = \qty{60}{\volt}$ is shown as
    the dahed line.}
    \label{fig:U_dep}
\end{figure}
By comparing this plot with the theoretical prediction \autoref{fig:leakage_current}, the depletion voltage $U_{\mathrm{dep}}$ is determined by identifying a kink
in the graph. In the case of this data, this is particularly difficult. Reasons for this are further adressed in \autoref{sec:Discussion}. The depletion voltage
estimated from the characteristic is $U_\mathrm{dep} = \qty{60}{\volt}$.\\
In the following, a bias voltage of \qty{80}{\volt} is used to ensure that the detector is depleted entierly.

\subsection{Pedestals, noise and common mode shift}
Like all electronic devices, silicon strip detectors are prone to electronic noise. Following \autoref{eq:ADC}, the ADC counts consist not only of the signal
but also the pedestal $P(i,k)$, the common mode shift $D(k)$ and the noise $N(i)$. 
For the determination of these quantities, the ADC counts for the strips are measured and used to calculate the target values.
The counts of each individual strip without external signal is called the pedestal $P(i,k)$ and is calculated via \autoref{eq:pedestals}. A graphic representation
is given in \autoref{fig:pedestals}. 
A global disturbance to all of the strips is refered to as the common mode shift $D(k)$ and determined by averaging over the difference of the ADC counts and
the pedestals (see \autoref{eq:common_noise}). This is visualized in \autoref{fig:common_noise}.
Taking the root-mean-square of the ADC counts subtracted of the pedestal and common mode shift yields the noise $N(i)$ of each pixel. With the help of \autoref{eq:noise},
the data are represented in \autoref{fig:noise}.
\begin{figure}
    \centering
    \begin{subfigure}{0.8\textwidth}
      \includegraphics[width = \textwidth]{build/pedestal.pdf}
      \caption{Pedestal $P(i,k)$.}
      \label{fig:pedestals}
    \end{subfigure}
    \hfill
    \begin{subfigure}{0.8\textwidth}
      \includegraphics[width = \textwidth]{common_mode_shift.pdf}
      \caption{Common mode shift $D(k)$.}
      \label{fig:common_noise}
    \end{subfigure}
    \vfill
    \begin{subfigure}{0.8\textwidth}
        \includegraphics[width = \textwidth]{noise.pdf}
        \caption{Noise $N(i)$.}
        \label{fig:noise}
      \end{subfigure}
    \caption{Visualization representation of the pedestal $P(i,k)$, common mode shift~$D(k)$ and noise $N(i)$.}
    \label{fig:pedestal_etc}
  \end{figure}

Analyzing the common mode shift $D(k)$ in \autoref{fig:common_noise}, it is clear that the data follow a Gaussian distribution. For the noise $N(i)$, a uniform 
distribution with increasing noise at the sensors' edges is present.

\subsection{Calibration measurements}
\label{sec:analysis_calib}
A proper calibration of the system is necessary to ensure a good quality of the data. Therefore, the delay time is optimized by performing a delay scan and determining
the maximum value and entering this value into the GUI of the program.\footnote{The data to this are unfortunately lost because of the complications we had with the setup.
A similar calibration is performed for the laser synchronization run in \autoref{sec:characteristics_strip_sensor}.}\\
Moreover, a calibration run for five different channels is carried out. For this, a known charge is injected and the ADC counts are measured. Additionally, one
calibration run for a bias voltage of $U_{\mathrm{bias}}=\qty{0}{\volt}$ is executed. In the diagrams of \autoref{fig:calib}, the calibration runs are depicted.

\begin{figure}
    \centering
    \begin{subfigure}{0.6\textwidth}
        \includegraphics[width = \textwidth]{build/calib_all_channels.pdf}
        \caption{Individual channels}
        \label{fig:calib_all_channels}
    \end{subfigure}
    \hfill
    \begin{subfigure}{0.6\textwidth}
        \includegraphics[width = \textwidth]{build/calib_mean.pdf}
        \caption{Mean values and $U_{\mathrm{bias}}=\qty{0}{\volt}$.}
        \label{fig:calib_mean}
    \end{subfigure}
    \caption{Calibration run: dependence of the ADC counts on the injected charge.}
    \label{fig:calib}
\end{figure}

The measurements of the five different channels are mostly in agreement with each other while the $U_{\mathrm{bias}}=\qty{0}{\volt}$ run shows slightly
less ADC counts. \\
In addition to these plots, the injected charge is also plotted against the measured ADC counts and fitted with a fourth-degree polynominal 
function. This is performed in \autoref{fig:calib_polyfit}. With the help of the \texttt{python} extension \texttt{scipy}~\cite{scipy}, the parameters for
a polynominal of the form $f(x) = a + bx + cx^2 + dx^3 + ex^4$ are extracted as
\begin{align*}
    a &= \qty{1.2(0.4)e3}{\per\raiseto{0}\elementarycharge} \\
    b &= \qty{151+-19}{\per\raiseto{1}\elementarycharge} \\
    c &= \qty{3.28+-0.29}{\per\raiseto{2}\elementarycharge} \\
    d &= \qty{-0.0235+-0.0017}{\per\raiseto{3}\elementarycharge} \\
    e &= \qty{6.34(0.32)e5}{\per\raiseto{4}\elementarycharge}. \\
\end{align*} 
Later in \autoref{sec:large_q}, this function is used to convert the ADC counts to a corresponding energy.

\subsection{Characterisics of the strip sensor}
\label{sec:characteristics_strip_sensor}
Among other variables, the pitch and the width of the strips are of high interest. Theses characteristics can be determined by exciting the sensor with a laser.
Before theses measurements can be conducted, the delay between the laser signal and the chip readout needs to be optimized. For that reason, a laser synchronization
run is performed where the delay time is varied and the ADC counts are measured. Such a run is shown in \autoref{fig:laser_sync}. The delay time at which the ADC
counts are maximized is evaluated as \qty{64}{\nano\second} and entered into the GUI of the program.
\begin{figure}
    \centering
    \includegraphics[width=0.8\textwidth]{build/laser_delay.pdf}
    \caption{Graphical representation of the laser synchronization run with the ADC counts depending on the delay time $t$.}
    \label{fig:laser_sync}
\end{figure}

With the laser synchronized to the chip readout, the actual measurements of the strip sensor characterisics can begin. Therefore, the laser is moved 
in \qty{10}{\micro\metre} intervals along the sensor. The ADC counts are measured and represented graphically in \autoref{fig:laser_scan}.
\begin{figure}
    \centering
    \begin{subfigure}{0.49\textwidth}
        \includegraphics[width = \textwidth]{build/laser_scan.pdf}
        \caption{All channels.}
        \label{fig:laser_scan_all_channels}
    \end{subfigure}
    \hfill
    \begin{subfigure}{0.49\textwidth}
        \includegraphics[width = \textwidth]{build/laser_scan_mag.pdf}
        \caption{Magnification of relevant channels.}
        \label{fig:laser_scan_mag}
    \end{subfigure}
    \caption{Laser scan: the ADC counts are depicted for each channel and distance along the axis where the laser is moved.}
    \label{fig:laser_scan}
\end{figure}

It is clear that the laser is focused on the strips with the channels 83 and 84. The ADC counts of these channels are further considered for the determination of 
the chips' characterisics (see \autoref{fig:laser_channel_83_84}). The pitch of the sensor can be measured by calculating the distance between the maxima of the ADC counts. The width of the strips follows
from the width of the peaks. 
\begin{figure}
    \centering
    \begin{subfigure}{0.49\textwidth}
        \includegraphics[width = \textwidth]{build/laser_channel_83.pdf}
        \caption{Channel 83.}
        \label{fig:laser_channel_83}
    \end{subfigure}
    \hfill
    \begin{subfigure}{0.49\textwidth}
        \includegraphics[width = \textwidth]{build/laser_channel_84.pdf}
        \caption{Channel 84}
        \label{fig:laser_channel_84}
    \end{subfigure}
    \caption{Depiction of the ADC counts and distance of the laser for the relevant channels 83 and 84.}
    \label{fig:laser_channel_83_84}
\end{figure}

\begin{table}
    \centering
    \caption{Results for the measurements of the pitch and strip width extracted from \autoref{fig:laser_channel_83_84}. All values in \unit{\micro\metre}.}
    \label{tab:pitch_width}
    \begin{tabular}{l | c c | c || c c | c || c c | c}
      \toprule
      {} & {Start 1} & {End 1} & {Width} & {Start 2} & {End 2} & {Width} & {Max 1} & {Max 2} & {Pitch} \\
      \midrule
      {Channel 83} & 120 & 220 & 100    & 240 & 240 & 100   & 170 & 290 & 120 \\ 
      {Channel 84} &  -  &  60 &  -     &  80 & 180 & 100   &  20 & 130 & 110 \\
      \bottomrule
    \end{tabular}
  \end{table}


\subsection{Charge collection efficiency}
In this part of the analysis, the charge collection efficiency (CCE) is determined in two manners. First, the laser is used to excite the sensor where a $\beta^-$ 
source is employed later.

\subsubsection{CCEL}
With the help of a laser, that is sharply focused on one strip of the sensor, the CCE can be delibereated. Again, the ADC counts are recorded but this time the
bias voltage is varied. Similar to the laser scan, the results are shown in a colormap, see \autoref{fig:ccel}.
\begin{figure}
    \centering
    \begin{subfigure}{0.49\textwidth}
        \includegraphics[width = \textwidth]{build/ccel.pdf}
        \caption{All channels.}
        \label{fig:ccel_all_channels}
    \end{subfigure}
    \hfill
    \begin{subfigure}{0.49\textwidth}
        \includegraphics[width = \textwidth]{build/ccel_mag.pdf}
        \caption{Magnification of relevant channels.}
        \label{fig:ccel_mag}
    \end{subfigure}
    \caption{CCE laser scan: the ADC counts are depicted for each channel and applied bias voltage.}
    \label{fig:ccel}
\end{figure}

The penetration depth of the laser can be reconstructed by fitting \autoref{eq:CCE} to data of the relevant channel from \autoref{fig:ccel}.
Therefore, the ADC counts need to be normalized to their maximum. In theory, the ADC counts reach a plateau for values of the bias voltage greater 
than the depletion voltage. Because of imperfections, the measured plateau is not strictly constant. As a solution to this, the values are normalized to the
mean value of the plateau. \\
The free parameter $a$ from \autoref{eq:CCE} is identified as the penetration depth and determined as
\begin{equation*}
    a = \qty{212(24)}{\micro\metre}.
\end{equation*}

\begin{figure}
    \centering
    \includegraphics[width=0.8\textwidth]{build/ccel_channel_82.pdf}
    \caption{Laser CCE scan: normalized ADC counts for different bias voltages.}
    \label{fig:ccel_channel_82}
\end{figure}

\subsubsection{CCEQ}
Analogous to the previous analysis step, the measurement of the CCE with a $\beta^-$ source yields a plot of the ADC counts in dependence of the bias voltage.
This plot, represented in \autoref{fig:ccel_channel_82} is similar to the laser CCE scan (\autoref{fig:ccel_channel_82}). For bias voltages greater than
the depletion voltage of $U_\mathrm{dep} = \qty{80}{\volt}$ the ADC counts are mostly constant.
\begin{figure}
    \centering
    \includegraphics[width=0.8\textwidth]{build/cceq.pdf}
    \caption{Source CCE scan: ADC counts for different bias voltages.}
    \label{fig:cceq}
\end{figure}

\subsection{Large source scan}
\label{sec:large_q}
In the final part of the measurement series, the sensor is exposed to the source for a total of approximately \num{1000000} events. In \autoref{fig:large_q_scan}, 
the data for this large source scan are represented. \autoref{fig:no_cluster} shows the number of clusters that each event caused. The number of responding channels
to each cluster ist depicted in \autoref{fig:no_channels}. Here, no more than five different channels were triggered. The distribution of all the counts on all the
channels is given in \autoref{fig:large_q_hitmap}. The shape of the distribution suggests that the source was focused at the strips corresponding to the channels
70 to 90.
\begin{figure}
    \centering
    \begin{subfigure}{0.49\textwidth}
      \includegraphics[width = \textwidth]{build/number_cluster.pdf}
      \caption{Number of clusters.}
      \label{fig:no_cluster}
    \end{subfigure}
    \hfill
    \begin{subfigure}{0.49\textwidth}
      \includegraphics[width = \textwidth]{build/number_channels.pdf}
      \caption{Number of activated channels per cluster.}
      \label{fig:no_channels}
    \end{subfigure}
    \vfill
    \begin{subfigure}{0.49\textwidth}
        \includegraphics[width = \textwidth]{build/hitmap.pdf}
        \caption{Hitmap, events per channel.}
        \label{fig:large_q_hitmap}
      \end{subfigure}
    \caption{Visualization of different aspects of the large source scan.}
    \label{fig:large_q_scan}
  \end{figure}

  When binning the ADC counts for this large source scan, a maximum can be determined. With the help of the previously calculated correlation function of the 
  ADC counts and charge (see \autoref{sec:analysis_calib}), the energy spectrum can be extracted. Therefore, the charge needed to create electron-hole pairs of
  \qty{3.6}{\electronvolt} needs to be accounted for. The resulting distribution is displayed in \autoref{fig:large_q_energy}.

  \begin{figure}
    \centering
    \begin{subfigure}{0.49\textwidth}
        \includegraphics[width = \textwidth]{build/cluster_adc.pdf}
        \caption{ADC counts}
        \label{fig:large_q_adc}
    \end{subfigure}
    \hfill
    \begin{subfigure}{0.49\textwidth}
        \includegraphics[width = \textwidth]{build/cluster_adc_energy.pdf}
        \caption{Energy}
        \label{fig:large_q_energy}
    \end{subfigure}
    \caption{Spreading of the ADC counts and energy.}
    \label{fig:large_q_adc_energy}
\end{figure}

The mean value of the energy as well as the most probable value are read off as
\begin{align*}
    E_{\mathrm{mean}} &= \qty{135.94}{\kilo\electronvolt} \\
    E_{\mathrm{MPV}}  &= \qty{89}{\kilo\electronvolt}. \\
\end{align*}

% Optimal delay laser scan: 64.0
% Laserscan shape adc: (128, 35)
% Mean of plateau (CCE): 132.47836484615385
% -----------------
% Params CCEL
% Parameter 0: 212+/-24 um
% -----------------
% ------ CCEQ ------
% E_mean = 135.93623449798488 keV
% E_MPV = 89 keV